\documentclass[a4paper,12pt]{scrartcl} 

%\usepackage[latin1]{inputenc} 
%Apple \usepackage[applemac]{inputenc} 
\usepackage[utf8]{inputenc}
\usepackage[ngerman]{babel}
\usepackage[T1]{fontenc}

%Das Paket erzeugt ein anklickbares Verzeichnis in der PDF-Datei.
\usepackage[hyperfootnotes=false,colorlinks=true,linkcolor=black,urlcolor=black]{hyperref}

%Das Paket wird fr die anderthalb-zeiligen Zeilenabstand bentigt
\usepackage{setspace}

%Einrückung eines neuen Absatzes
\setlength{\parindent}{0em}

%Definition der Rnder
\usepackage[paper=a4paper,left=30mm,right=30mm,top=30mm,bottom=30mm]{geometry} 

\usepackage{amsfonts}
\usepackage{amsmath}
\usepackage{cancel}
\usepackage{graphicx}
\usepackage{mathcomp}
\usepackage{polynom}
\usepackage{footnote}

%Abstand der Fußnoten
\deffootnote{1em}{1em}{\textsuperscript{\thefootnotemark\ }}

%Regeln, bis zu welcher Tiefe (section,subsection,subsubsection) Überschriften angezeigt werden sollen (Anzeige der Überschriften im Verzeichnis / Anzeige der Nummerierung)
\setcounter{tocdepth}{3}
\setcounter{secnumdepth}{3}

%-------------------
%Ende des Kopfbereiches
%-------------------


\begin{document}

%Beginn der Titelseite
\begin{titlepage}
\begin{small}
\vfill {AKAD\\ 
Bachelor of Science (Wirtschaftsinformatik) \\ 
Modulzusammenfassung}
\end{small}


\begin{center}
\begin{Large}
\vfill {\textsf{\textbf{
UFU06 \\
\vspace*{1cm} 
Formelsammlung
}}}
\end{Large}
\end{center}

\begin{small}
\vfill Daniel Falkner \\ Rotbach 529 \\  94078 Freyung \\  daniel.falkner@akad.de \\ 
\today
\end{small}

\end{titlepage}
%Ende der Titelseite


%Inhaltsverzeichnis (aktualisiert sich erst nach dem zweiten Setzen)
\tableofcontents
\thispagestyle{empty}

%Beginn einer neuen Seite
\clearpage

%Anderthalbzeiliger Zeilenabstand ab hier
\onehalfspacing

\pagestyle{plain}


\section{Entscheidungen bei Unsicherheit}
\subsection{Entscheidungen bei Risiko}
\subsubsection{Die Bayes-Regel}
$ \mu = \sum\limits_{i=1}^n e_{ij} * p_j $
\subsubsection{Das ($\mu, \sigma$)-Prinzip}
$ \sigma_i = \sqrt{\sum\limits_{j=1}^n p_j*(e_{ij} - \mu_i)^2}$

\subsection{Entscheidungen bei Unwissenheit}
\subsubsection{Die Minimax-Regel}
Zeilenminimum suchen und höchstes Ergebnis verwenden.
\subsubsection{Die Maximax-Regel}
Zeilenmaximum suchen und höchstes Ergebnis verwenden.
\subsubsection{Die Hurwicz-Regel}
Zeilenminimum und -maximum ermitteln. \\
$N (Nutzen) = \lambda * Zeilenmaximum + (1 - \lambda) * Zeilenminimum $
\subsubsection{Die Savange-Niehans-Regel}
Regel des kleinsten Bedauerns \\ 
\begin{itemize}
\item Spaltenmaximum ermitteln
\item Neue Spalte ist Spaltenmaximum minus Spalte
\item Zeilenmaximum und das niedrigste Ergebnis verwenden.
\end{itemize}
\subsubsection{Die Laplace-Regel}
Summe der Ergebnisse durch Anzahl der Umweltzustände. Das höchste Ergebnis verwenden.

\subsection{Entscheidungen in Spielsituationen}
\subsubsection{Zweipersonen-Nullsummenspiele}
\begin{itemize}
\item Zeilenminimum mit Kreis
\item Spaltenmaximum mit Quadrat
\item Schnittpunkt ist Sattelpunkt (Gleichgewichtsstrategien)
\end{itemize}

\section{Papiercomputer}
\subsection{Beziehungen}
\begin{itemize}
\item 3 = starke, überproportionale Beziehung
\item 2 = mittlere, etwa proportionale Beziehung
\item 1 = sehr schwache Beziehung
\item 0 = keine Beziehung
\end{itemize}

\subsection{zentrale Größen im Problemgeflecht}
\begin{itemize}
\item Die Zeilensumme einer Größe gibt an, wie stark sie auf andere Größen des Netzwerks wirkt (= Aktivsumme)
\item Die Spaltensumme dagegen lässt erkennen, wie sensibel die betreffende Größe für Veränderungen aus dem übrigen System ist (= Passivsumme)
\end{itemize}

\subsection{Schlüsselrollen}
\begin{itemize}
\item Aktive Größen haben eine hohe Aktivsumme, aber eine niedrige Passivsumme. Sie beeinflussen andere stark, stehen aber selbst unter geringem Einfluss durch andere (Aktive Elemente links oben)
\item  Im Gegensatz dazu werden Größen, die eine hohe Passivsumme, aber nur eine geringe Aktivsumme haben, als reaktive Größen bezeichnet. (Reaktive Elemente rechts unten)
\item Träge oder puffernd werden Größen genannt, die andere nur schwach beeinflussen, selbst aber auch nur gering beeinflusst werden. (Träge Elemente links unten)
\item  Kritische Größen schließlich üben starken Einfluss auf andere aus, stehen aber selbst auch unter starkem Einfluss durch andere. (Kritische Elemente rechts oben)
\end{itemize}

\subsection{Nachteile des Papiercomputers}
\begin{itemize}
\item In manchen Fällen ist der paarweise Größenvergleich bei der Erstellung der Einflussmatrix der Komplexität der Problemsituation unter Umständen nicht angemessen.
\item Durch die isolierte Betrachtung kann es zu Mehrfachzählungen von Wirkungen kommen, die das Gesamtbild verfälschen.
\item Ferner kann die numerische, zahlenmäßige Erfassung der Zusammenhänge den beteiligten Personen eine Scheinobjektivität vorgaukeln.
\end{itemize}

\section{Netzplan}
\subsection{Feldeinteilung}
\begin{table}[ht]
\centering
\begin{tabular}{|c|c|c|}
\hline
  $i$ & $FAZ_i$ & $FEZ_i$ \\
\hline  
  $d_i$ & $SAZ_i$ & $SEZ_i$ \\
\hline
  $GP_i$ & $FP_i$ & $UP_i$ \\
\hline
 \end{tabular}
\end{table}%
 
\subsection{Regeln}
\begin{itemize}
\item $i$ \footnote{Vorgangsbezeichnung} (Name oder Nummer, i) 
\item $d_i$ \footnote{Vorgangsdauer} (Zeit)
\item $FEZ_i = FAZ_i + d_i$ \footnote{frühester Endtermin}
\item $FAZ_{i+1} = FEZ_i$ \footnote{frühester Anfangszeitpunkt, beim ersten Vorgang 0, bei mehreren Vorgängern der größte Wert}
\item $SEZ_n = FEZ_n$ \footnote{späteste Endzeitpunkt beim letzten Vorgang}
\item $SAZ_i = SEZ_i - d_i$ \footnote{späteste Anfangszeitpunkt, bei mehreren Nachfolgern der kleinste Wert}
\item $GP_i = SEZ_i - FEZ_i$ oder $GP_i = SAZ_i - FAZ_i$ \footnote{gesamte Puffer}
\item $FP_i = FAZ (= Nachfolger) - FEZ (= betrachteter Vorgang)$ \footnote{freie Puffer, bei mehreren Vorgängern der kleinste Wert, bei mehreren Nachfolgern der kleinste Wert}
\item $UP_i = FAZ (= Nachfolger) - SEZ (= Vorg"anger) - d_i$ \footnote{unabhängige Puffer, bei mehreren Vorgängern der größten Wert, bei mehreren Nachfolgern den kleinsten Wert}
\item $GP_i = FP_i = UP_i = 0$ (Kritische Vörgänge -> es besteht ein kritischer Pfad)

\end{itemize}










\end{document}

