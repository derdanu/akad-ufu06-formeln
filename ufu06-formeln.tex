\documentclass[a4paper,12pt]{scrartcl} 

%\usepackage[latin1]{inputenc} 
%Apple \usepackage[applemac]{inputenc} 
\usepackage[utf8]{inputenc}
\usepackage[ngerman]{babel}
\usepackage[T1]{fontenc}

%Das Paket erzeugt ein anklickbares Verzeichnis in der PDF-Datei.
\usepackage[hyperfootnotes=false,colorlinks=true,linkcolor=black,urlcolor=black]{hyperref}

%Das Paket wird fr die anderthalb-zeiligen Zeilenabstand bentigt
\usepackage{setspace}

%Einrückung eines neuen Absatzes
\setlength{\parindent}{0em}

%Definition der Rnder
\usepackage[paper=a4paper,left=30mm,right=30mm,top=30mm,bottom=30mm]{geometry} 

\usepackage{amsfonts}
\usepackage{amsmath}
\usepackage{cancel}
\usepackage{graphicx}
\usepackage{mathcomp}
\usepackage{polynom}


%Abstand der Fußnoten
\deffootnote{1em}{1em}{\textsuperscript{\thefootnotemark\ }}

%Regeln, bis zu welcher Tiefe (section,subsection,subsubsection) Überschriften angezeigt werden sollen (Anzeige der Überschriften im Verzeichnis / Anzeige der Nummerierung)
\setcounter{tocdepth}{3}
\setcounter{secnumdepth}{3}

%-------------------
%Ende des Kopfbereiches
%-------------------


\begin{document}

%Beginn der Titelseite
\begin{titlepage}
\begin{small}
\vfill {AKAD\\ 
Bachelor of Science (Wirtschaftsinformatik) \\ 
Modulzusammenfassung}
\end{small}


\begin{center}
\begin{Large}
\vfill {\textsf{\textbf{
UFU06 \\
\vspace*{1cm} 
Formelsammlung
}}}
\end{Large}
\end{center}

\begin{small}
\vfill Daniel Falkner \\ Rotbach 529 \\  94078 Freyung \\  daniel.falkner@akad.de \\ 
\today
\end{small}

\end{titlepage}
%Ende der Titelseite


%Inhaltsverzeichnis (aktualisiert sich erst nach dem zweiten Setzen)
\tableofcontents
\thispagestyle{empty}

%Beginn einer neuen Seite
\clearpage

%Anderthalbzeiliger Zeilenabstand ab hier
\onehalfspacing

\pagestyle{plain}


\section{Formeln}






\end{document}

